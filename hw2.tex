\documentclass[10pt,a4paper]{article}
\addtolength{\oddsidemargin}{-.875in}
\addtolength{\evensidemargin}{-.875in}
\addtolength{\textwidth}{1.75in}
\addtolength{\topmargin}{-.875in}
\addtolength{\textheight}{1.75in}

\usepackage{amsmath,amssymb}
\DeclareMathOperator*{\E}{\mathbb{E}}
\DeclareMathOperator*{\R}{\mathbb{R}}
\DeclareMathOperator*{\Q}{\mathbb{Q}}
\DeclareMathOperator*{\N}{\mathbb{N}}
\DeclareMathOperator*{\I}{\mathbb{I}}
\DeclareMathOperator*{\argmax}{arg\,max}
\DeclareMathOperator*{\argmin}{arg\,min}
\usepackage{mathtools}
\DeclarePairedDelimiter{\ceil}{\lceil}{\rceil}
\DeclarePairedDelimiter{\abs}{\lvert}{\rvert}
\DeclarePairedDelimiter{\l2}{\lVert}{\rVert}
\newcommand\m[1]{\begin{bmatrix}#1\end{bmatrix}} 
\begin{document}

Jesse Cai

MATH 164 Homework 2

304634445

\begin{enumerate}
    \item \textbf{9.3 Consider minimizing $f(x) = x^{\frac{4}{3}}$}
    \begin{enumerate}
        \item We can find $\nabla f(x) = g(x) = \frac{4}{3}\sqrt[3]x$ and $f"(x) =  \frac{4}{9} x^{- \frac{2}{3}}$
        
        $$x^{k+1} = x^k - F(x^k)^{-1}g^k = x^k - (\frac{4}{9}x^{-\frac{2}{3}})^{-1} \frac{4}{3} x^{\frac{1}{3}} = x^k - 3x^k = -2x^k$$
        
        \item We can expand the fomula to get $x^k = -2^k x^0$ which does not converge to 0 except when $x^0 = 0$.
    \end{enumerate}

    \item \textbf{9.4 Consider minimizing $f(x) = 100 (x_2 - x_1^2)^2 + (1 - x_1)^2$}
    \begin{enumerate}
        \item Note $f(x) \geq 0$, so the minima occurs when $f(x) = 0$. For this to be true, we can see that both $(1-x_1)^2$ and $(100(x_2 - x_1^2)^2$ are 0. 
        Solving the first equation yields $x_1 = 1$ and plugging this into the second part and solving yields $x_2 = 1$. So the unique minima is $[1,1]$.

        \item $\nabla f(x) = [400x_1^3 - 400x_1x_2 + 2x_1 -2, 200(x_2-x_1^2)]^T$
        $$F(x) = \begin{bmatrix} 1200x_1^2 - 400x_2 +2 & -400x_1 \\ -400x_1 & 200 \end{bmatrix}$$        
        $$F(x)^{-1} = \frac{1}{80000(x_1^2 -x_2) + 400} \begin{bmatrix} 200 & 400x_1 \\400x_1 & 1200x_1^2 - 400x_2 +2 \end{bmatrix}$$        

        $x_0 = [0, 0]^T$, $g_0 = [-2, 0]$, $x_1 = x_0 - F(x^0)^{-1} g^0  =  - \frac{1}{400} \begin{bmatrix} 200 & 0 \\ 0 & 2 \end{bmatrix} \begin{bmatrix} -2 \\ 0 \end{bmatrix} = \begin{bmatrix}1 \\0 \end{bmatrix}$

        $x_1 = [1, 0]^T$, $g_1 = [400, -200]$, $x_2 = x_1 - F(x^1)^{-1} g^1  = \begin{bmatrix} 1\\ 0 \end{bmatrix} - \frac{1}{80400} \begin{bmatrix} 200 & 400 \\ 400 & 1202 \end{bmatrix} \begin{bmatrix} 400 \\ -200 \end{bmatrix} = \begin{bmatrix}1 \\1 \end{bmatrix}$
        \item $x_0 = [0, 0]^T$, $g_0 = [-2, 0]$, $x_1 = x_0 - 0.05 g_0 = [0.1, 0]^T$

        $x_1 = [0.1, 0]^T$, $g_1 = [-1.4, 2]$, $x_2 = x_1 - 0.05 g_1 = [0.17, 0.1]^T$

    \end{enumerate}

    \item \textbf{10.1 Show that $d^0 \dots d^{n-1}$ are $Q$ conjugate}
    
    We can do this via induction. Let $P(n) = \forall i < n : d_{i}^TQd_n = 0$.
   
    Base Case: $d_0 = p_0$ and $d_1 = p_1 - \frac{p_1^TQd_0}{d_0^TQd_0}d_0$

    $$d_0^TQd_1 = p_0^TQ(p_1 - \frac{p_1^TQp_0}{p_0^TQp_0}p_0) = p_0^TQp_1 - \frac{p_1^TQp_0}{p_0^TQp_0}p_0^TQp_0 = (p_1^TQp_0) (1-1) = 0$$

    Inductive Step: Assume $P(n)$ holds, then we want to show $P(n) \implies P(n+1)$.

    $$d_{k+1} = p_{k+1} - \sum_{i=0}^k \frac{p_{k+1}^TQd_i}{d_i^TQd_i}d_i \implies d_j^TQd_{k+1} = d_j^TQp_{k+1} - \sum_{i=0}^k \frac{p_{k+1}^TQd_i}{d_i^TQd_i}d_i$$

    But we know by $P(n)$ that $ i \neq j \implies d_j^Td_i = 0$ and then by symmetry of $Q$.

    $$d_j^TQd_{k+1} = d_j^TQp_{k+1} - \frac{p_{k+1}^TQd_j}{d_j^TQd_j}d_j^TQd_j = d_j^TQp_{k+1} - p_{k+1}^TQd_j = 0$$


    \item \textbf{10.2 Show that if $g_k^Td_k \neq 0$ then $\{ d_0 \ldots d_{n-1} \}$ are Q-conjugate.}
    
    $$f(x_k + \alpha d_k) = \frac{1}{2} (x_k + \alpha d_k)^T Q (x_k + \alpha d_k) - (x_k + \alpha d_k)^Tb= \frac{1}{2} d_k^TQd_k \alpha^2 + g_k^Td_k \alpha + C$$
    
    We can take the derivitave of $\phi$ with respect to $\alpha$, which is 0 due to Lemma 10.2.

    $\phi'(\alpha_k) = \nabla f(x_k + \alpha_k d_k)^Td_k = g_{k+1}^td_k = 0$

    But then since $g_k^Td_i = 0$:

    $$d_k^TQd_i = \frac{1}{\alpha_k}(x_{k+1} - x_k)^TQd_i = \frac{1}{\alpha_k}(g_{k+1} - g_k)^TQd_i = 0$$

     
    \item \textbf{10.3 Show that in the conjugate gradient method for a standard quadratic $d_k^TQd_k = - d_k^TQg_k$}
    
    Again we can do this via induction. Let $P(k) = d_k^TQd_k = - d_k^TQg_k$

    Base Case: By definition $g_0  = - d_0 \implies d_0^TQd_0 = -d_0^Tg_0$ so $P(0)$ holds.

    Inductive Step: Assume $d_k^TQd_k = -d_k^TQg_k$. By definition $d_{k+1} = - g_{k+1} - \beta_k d_k$

    Then since $d_k, d_{k+1}$ are $Q$-conjugate:
    $$d_{k+1}^TQd_{k+1} = d_{k+1}^TQ(-g_{k+1} - \beta_k d_k) = -d_{k+1}^TQg_{k+1} - \beta_kd_{k+1}^TQd_k = -d_{k+1}^TQg_{k+1}$$

    
    
    \item \textbf{10.4 Let $Q$ be a real symmetric matrix}
    \begin{enumerate}
        \item Since $Q$ is a real symmetric matrix, there exists an orthogonal basis of eigenvectors $\{v_1 \ldots v_n\}$.
        This basis is $Q-conjugate$. 

        Pick $i, j$ s.t. $i \neq j$. Then because $v_j$ is an eigenvector and since $v_i, v_j$ are orthogonal.
        $$\implies v_i^TQv_j = v^i\lambda_jv_j = \lambda_j v_i^Tv_j = 0$$

        \item If $\{d_1 \ldots d_n\}$ is $Q$-conjugate and also orthogonal then.
        $$d_i^TQd_j = 0 = d_i^Td_j = \lambda_j d_i^Td_j = d_i^T \lambda_j d_j \implies Qd_j = \lambda_jd_j$$

        So $d_j$ must be an eigenvector.
    \end{enumerate}

    \item \textbf{10.5 For a standard quadratic with $d_{k+1} = \gamma_k g_{k+1}+d_k$ find an expression for $\gamma_k$}
    
    Recall that our original definition of the conjugate gradient method gives $d_{k+1} = -g_{k+1} + \beta_k d_k$.

    So we can plug in and solve: $$\gamma_k g_{k+1}+d_k =  -g_{k+1} + \beta_k d_k \implies \gamma_k = \frac{-1}{\beta_k} = \frac{d_k^TQd_k}{g_{k+1}^TQd_k}$$


    \item \textbf{10.6 Suppose we are minimizing a standard quadratic with update rule $d_{k+1} = \alpha_k g_{k+1} +\beta_k d_, \alpha_k, \beta_k \in \mathbb{R}$. }
    \begin{enumerate}
        \item Show that $d_k \in V_{k+1}$ and $x_k \in V_k$. 
        We can prove this by induction. Let $P(k) = d_k \in V_{k+1} \land x_k \in V_k$.

        Base Case $P(0)$:

        $d_0 = a_0g_0 = -a_0b \in V_1$ since $b \in V_1$ 
        $x_0 = 0 \in V_0$ as $0$ is in every subspace.

        Inductive Step: Assume $d_k \in V_{k+1} \land x_k \in V_{k}$.

        Note $V_k \subset V_{k+1} \implies x_k \in V_{k+1}$ so 

        $x_{k+1} = x_k + \alpha_k d_k \implies x_{k+1} \in V_{k+1}$

        $d_{k+1} = a_k g_{k+1} + b_k d_k$. But $Qx_{k+1} -b \in V_{k+2}$ by definition so $d_{k+1} \in V_{k+2}$

        \item The conjugate gradient algorithm finds the min of each subspace $V_k$ along each step.

    \end{enumerate}

    \item \textbf{10.7 Consider a standard quadratic function $f(x)$ and $\phi(a) = f(x_0+Da)$ where $D$ is a matrix of rank $r$. Show $\phi$ is a standard quadratic }
    
    Note $f(x) = \frac{1}{2}x^TQx - x^Tb +c$

    $$\phi(a) = \frac{1}{2}(x_0+Da)^TQ(x_0+Da) - (x_0+Da)^Tb = x_0^TQx_0 + (aD)^TQDa - x_0^Tb - (Da)^Tb$$

    Using the fact that $x_0$ and $b$ are constants, and then $(Da)^T = a^TD^T$ we get:

    $$\phi(a) = (Da)^TQDa - (Da)^Tb +c = aD^TQDa = a^TDb \implies Q' = D^TQD, b' = Db$$

    So $\phi(a)$ can be written in the standard quadratic form.

    \item \textbf{10.8 Consider a conjugate gradient algorithm applied to a quadratic function}
    \begin{enumerate}    
        \item WTS $\forall 0 \leq k \leq n-1 \land 0 \leq i \leq k: g_{k+1}^Tg_i = 0 $.
        
        Recall that $d_{k+1} = -g_{k+1} + \beta_k d_k \implies  g_{k+1} = \beta_k d_k - d_{k+1}$.

        $$g_{k+1}^Tg_i = (\beta_k d_k -d_{k+1})^T g_i$$ and by Lemma 10.2 $d_k^Tg_i$ and $d_{k+1}^Tg_i = 0$

        \item $g_{k+1}^TQg_i  = (\beta_k d_k - d_{k+1})^TQ(\beta_{i-1} d_{i-1} - d_i)$
        
        $$ = \beta_k\beta_{i-1} d_k^TQd_{i-1}   \beta_k d_k^TQd_{i} - \beta_{i-1} d_{k+1}^TQd_{i-1} + d_{k+1}^TQd_i$$

        Then by $Q$-conjuacy this is 0 so $g_{k+1}$ and $g_i$ are $Q$-conjugate. $rank(D) = r \implies Da = 0 \iff a =0 \implies Q'$ is positive definite.

    \end{enumerate}

    \item \textbf{10.9 Represent $f(x_1, x_2) = \frac{5}{2}x_1^2 +x_2^2 - 3x_1x_2 -x_2 -7$ in standard form and find $d_1$.}
    
    $$f(x) = \frac{1}{2}x^T\begin{bmatrix} 5 & -3 \\ -3 & 2\end{bmatrix}x - [0, 1]^Tx - 7$$
    
    Recall $d_0 = g_0 = Qx_0-b = Q(0) - b = [0, 1]^T$

    $x_1 = x_0 - \alpha_0 d_0 = -\frac{g_0^Td_0}{d_0^TQd_0} d_0=[0, \frac{1}{2}]^T$

    $g_1 = \nabla f(x_1) = Q[0, \frac{1}{2}] -b = [-\frac{3}{2}, 0]^T$

    $d_1 = -[-\frac{3}{2}, 0]^T + \frac{[-\frac{3}{2}, 0]^T[0, 1]}{2} [0,1]^T = [\frac{3}{2}, \frac{9}{4}]^T$

    \item \textbf{10.10 Consider minimizing $f(x_1, x_2) = \frac{5}{2}x_1^2 +\frac{1}{2}x_2^2 + 2x_1x_2 -3x_1 - x_2$.} 
    \begin{enumerate}
        \item $$f(x) = \frac{1}{2}x^T\begin{bmatrix} 5 & 2 \\ 2 & 1\end{bmatrix}x - [3, 1]^Tx$$
        \item $x_0 = 0, -d_0 = g_0 = [3, 1]^T, x_1 = -\frac{g_0^Tg_0}{g_0^TQg_0} d_0 = \frac{5}{29}[3, 1]^T$
        
        $g_1 = ,d_1 = - g_1+ \beta_0 d_0 $

        $x_2 = [1, -1]^T$.

        \item Solving this analyticaly, we get the same answeras above.
        $$x^* = Q^{-1}b = \begin{bmatrix} 1 & -2 \\ -2 & 5\end{bmatrix} \begin{bmatrix} 3 \\ 1\end{bmatrix} = \begin{bmatrix} 1 \\ -1 \end{bmatrix}$$
    \end{enumerate}

    \item \textbf{11.1}
    \begin{enumerate}
        \item $\phi(\alpha) = f(x_k + \alpha d_k) \implies  \phi'(\alpha) = d_k^Tg_k$. But $\phi'$ is continuous so if $d_k^Tg_k < 0 \implies \exists \alpha : \forall \alpha \in (0, \alpha) \phi(\alpha) > \phi(0)$ 
        \item We know that $alpha_k = \argmin_{\alpha \geq 0} \phi(\alpha)$ so either $alpha_k > 0$ or $\alpha_k = 0$, but $\alpha_k = 0 $ leads to a contradiction with a) so $alpha_k > 0$.
        \item $g_{k+1} = f'(x_k + \alpha d_k) = \phi'(alpha_k) = 0 $
        \item \begin{enumerate}
            \item $d_k = -g_k \implies d_k^Tg_k = -g_k^Tg_k = - \l2{g_k}^2 < 0$
            \item $d_k = -F(x_k)^{-1} g_k$. But $Q$ is positive definite $\implies Q^{-1}$ is positive definite.
            
            So $d_k^Tg_k = -g_k^TF(x_k^{-1]})g_k < 0$
            \item $d_k = -g_k + \beta_{k-1} d_{k-1}$
            So using Lemma 10.2 we get $d_k^Tg_k = -g_k^Tg_k + \beta_{k-1}d_{k-1}^Tg_k = - g_k^Tg_k <0$
            \item $d_k = -H_k g_k $ If $H_k > 0 \implies d_k^Tg_k =  - g_k^THg_k < 0 $
        \end{enumerate}
        \item $\alpha_k = -\frac{d_k^Tg_k}{d_k^TQd_k}$
    \end{enumerate}
    
    \item \textbf{11.3 Consider minimizing $\phi(\alpha) = f(x_k + \alpha d_k)$ where $f$ is a standard quadratic.}
    \begin{enumerate}
        \item $\phi'(\alpha) = 0 = d_k^Tf'(x+ \alpha d_k)$
        $$ = d_k^T(Q(x+\alpha d_k) - b) = d_k^T(Qx - b) + \alpha d_k^TQd_k$$

        Solving for $\alpha$ yields $\alpha  = \frac{d_k^Tg_k}{d_k^TQd_k}$

        \item we can expand $d_k^TQd_k$ as $g_k^TH_k^TQHg_k$ so we know that $H_k$ must be positive definite to ensure Q is as well.
    \end{enumerate}

    \item \textbf{11.5 Minimize $f(x) = \frac{1}{2} x^T \m{1 & 0 \\0 & 2} x + x^T\m{1 \\ -1}$.}
    
    $g_0 = Qx_0 -b = -b = [-1, 1]^T$ and $d_0 = -H_0g_0 = -g_0 = [1, -1]^T$

    $\alpha_0 = \frac{g_0^Td_0}{d_0^TQd_0} = \frac{2}{3}$

    $x_1 = x_0 + \frac{2}{3}[1, -1]^T = [\frac{2}{3}, -\frac{2}{3}]^T$

    $g_1 = Qx_1 -b = [-\frac{1}{3}, -\frac{1}{3}]^T$

    $\Delta x_0 - H_0 \Delta g_0 = \begin{bmatrix}  0-\frac{2}{3} \\ 0 +\frac{2}{3}\end{bmatrix} - \begin{bmatrix} 1 + \frac{1}{3} \\ -1 + \frac{1}{3} \end{bmatrix} = \begin{bmatrix}0, \frac{2}{3}\end{bmatrix}$

    $\Delta g_0^T(\Delta x_0 - H_0 \Delta g_0) =  - \frac{8}{9}$
    
    $$H_1 = H_0 + \frac{(\Delta x_0 - H_0 \Delta g_0)(\Delta x_0 - H_0 \Delta g_0)^T}{\Delta g_0^T(\Delta x_0 - H_0 \Delta g_0)} = I + -\frac{9}{8} \m{0 & 0 \\ 0 & \frac{9}{4}} = \m{1 & 0 \\0 & \frac{1}{2}}$$

    $d_1 = -H_1g_1 =  - \m{1 & 0 \\0 & \frac{1}{2}} \m{-\frac{1}{3} \\ -\frac{1}{3}} = \m{\frac{1}{3} \\ \frac{1}{6}}$

    $\alpha_1 = \frac{g_1^Td_1}{d_1^TQd_1} = 1$

    $x_2 = x_* = \m{1 \\ -\frac{1}{2}}$
    
    \item \textbf{11.7 Show that if $H_k > 0$ and $\Delta g_k^T(\Delta x_k - H_k \Delta g_k) > 0$ then $H_{k+1} > 0$.}
   
    Recall that $$H_{k+1} = H_k + \frac{(\Delta x_k - H_k \Delta g_k)(\Delta x_k - H_k \Delta g_k)^T}{g_k^T(\Delta x_k - H_k \Delta g_k)}$$. 

    $$x^TH_{k+1}x = x^TH_kx + \frac{x^T(\Delta x_k - H_k \Delta g_k)(\Delta x_k - H_k \Delta g_k)^Tx}{g_k^T(\Delta x_k - H_k \Delta g_k)}$$

    $$ = x^TH_kx + \frac{[x^T(\Delta x_k - H_k \Delta g_k)]^2}{g_k^T(\Delta x_k - H_k \Delta g_k)}$$

    $$ = > 0 + > 0$$ since there's a square and we know the denominator is $>0$ by assumption.

    \item \textbf{11.9 Use rank-one correction method to generate two $Q$-conjugate directions.}
    
    This is the exact same equation as 11.5, so we can take the two vectors generated there:
    $$d_0 = -H_0g_0 = -g_0 = [1, -1]^T$$
    $$d_1 = -H_1g_1 =  - \m{1 & 0 \\0 & \frac{1}{2}} \m{-\frac{1}{3} \\ -\frac{1}{3}} = \m{\frac{1}{3} \\ \frac{1}{6}}$$

    \item \textbf{11.10 Apply the rank-1 algorithm to $f(x) = \frac{1}{2} x^T\m{4 & 2 \\ 2 & 2} x - x^T\m{-1 \\1} $}
    
    $g_0 = Qx_0 -b = -b = [-1, 1]^T$ and $d_0 = -H_0g_0 = -g_0 = [1, -1]^T$

    $\alpha_0 = \frac{g_0^Td_0}{d_0^TQd_0} = \frac{2}{3}$

    $x_1 = x_0 + \frac{2}{3}[1, -1]^T = [\frac{2}{3}, -\frac{2}{3}]^T$

    $g_1 = Qx_1 -b = [\frac{4}{3}, 0]^T$

    $\Delta x_0 - H_0 \Delta g_0 = \begin{bmatrix}  \frac{2}{3} \\ -\frac{2}{3}\end{bmatrix} - \begin{bmatrix} \frac{4}{3} -1 \\ 0 +1 \end{bmatrix} = \begin{bmatrix}\frac{1}{3} \\ \frac{-5}{3}\end{bmatrix}$

    $\Delta g_0^T(\Delta x_0 - H_0 \Delta g_0) =  [\frac{1}{3}, 1]^T\m{\frac{1}{3} \\ -\frac{5}{3}} = -\frac{14}{9}$
    
    $$H_1 = H_0 + \frac{(\Delta x_0 - H_0 \Delta g_0)(\Delta x_0 - H_0 \Delta g_0)^T}{\Delta g_0^T(\Delta x_0 - H_0 \Delta g_0)} = \frac{1}{2} \m{1 & -1 \\ -1 & 1}$$

    $d_1 = -H_1g_1 =  - \frac{1}{2}\m{1 & -1 \\-1 & 1} \m{-\frac{1}{3} \\ -\frac{1}{3}} = 0$

    So this algorithm fails for this problem.

\end{enumerate}
\end{document}