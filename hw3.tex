\documentclass[10pt,a4paper]{article}
\addtolength{\oddsidemargin}{-.875in}
\addtolength{\evensidemargin}{-.875in}
\addtolength{\textwidth}{1.75in}
\addtolength{\topmargin}{-.875in}
\addtolength{\textheight}{1.75in}

\usepackage{amsmath,amssymb}
\DeclareMathOperator*{\E}{\mathbb{E}}
\DeclareMathOperator*{\R}{\mathbb{R}}
\DeclareMathOperator*{\Q}{\mathbb{Q}}
\DeclareMathOperator*{\N}{\mathbb{N}}
\DeclareMathOperator*{\I}{\mathbb{I}}
\DeclareMathOperator*{\argmax}{arg\,max}
\DeclareMathOperator*{\argmin}{arg\,min}
\usepackage{mathtools}
\DeclarePairedDelimiter{\ceil}{\lceil}{\rceil}
\DeclarePairedDelimiter{\abs}{\lvert}{\rvert}
\DeclarePairedDelimiter{\l2}{\lVert}{\rVert}
\newcommand\m[1]{\begin{bmatrix}#1\end{bmatrix}} 
\begin{document}

Jesse Cai

MATH 164 Homework 3

304634445

\begin{enumerate}
    \item[15.1] 

     Minimize $f(x_1, x_2) = -2x_1 -x_2$ such that 

    $x_1 -s_1 = 2 $

    $x_1 + x_2 +s_2 = 3$

    $x_1 + 2x_2 + s_3 = 5$

    $x_1, x_2, s_1, s_2, s_3 \geq 0$

    \item [15.3] We can rewrite the original problem as follows
    
    Minimize $\sum_{i=1}^n c_i(x_i^+ - x_i^-)$ such that
    $$A(x^+ - x^-) = b, x^+, x^- \geq 0$$
    However for it to be true standard form we'll need to rewrite $A$ and $c$ as $A' = [A, -A], c'= [c, c]$.

    \item [15.5]
        Let number of units shipped be $x_1, x_2, x_3, x_4$ respectively for AC, AD. BC, BD.

        Minimize $f(x) = x_1 + 2x_2 + 3x_3 + 4x_4 $ s.t.

        $x_1 + x_2 + s_1 = 70$
        $x_3 + x_4 + s_2 = 80$
        $x_1 + x_3 = 50$
        $x_2 + x_4 = 60$
    
    \item [15.7] Let $x_i$ be the weight of item $i$ used.

    Then our goal is to minimize the total cost $f(x) = 2x_1 + 4x_2 + x_3 + 2x_4$ such that. 

        $x_1 + x_2 + x_3 + x_4 = 1000$

        $3x_1 + 8x_2 + 16x_3 + 4x_4 = 10000$

        $6x_1 + 46x_2 + 9x_3 + 9x_4 = 2000$

        $20x_1 + 5x_2 + 4x_3  = 5000$

    However the only solution is $\m{179 \\ -175 \\ 573 \\ 422}$ which is infeasible, so there is no solution.
    
    \item [15.9] The matrix has full rank $3$, so there are ${5 \choose 3}  = 10$ basic solutions.
    
        They are as follows:

        $ x^*_{1, 2, 3} = \frac{1}{17}\m{-4 & -80 & 83 & 0 & 0 }$

        $ x^*_{1, 2, 4} = \m{-10 & 49 & 0 & -83 & 0 }$

        $ x^*_{1, 2, 5} = \frac{1}{31}\m{105 & 25 & 0 & 0 & 83 }$

        $ x^*_{1, 3, 4} = \frac{1}{11}\m{-12 & 0 & 49 & -80 & 0 }$

        $ x^*_{1, 3, 5} = \frac{1}{35}\m{100 & 0 & 25 & 0 & 80 }$

        $ x^*_{1, 4, 5} = \frac{1}{18}\m{65 & 0 & 0 & 25 & 49 }$

        $ x^*_{2, 3, 4} = \m{0 & -6 & 5 & 2 & 0 }$

        $ x^*_{2, 3, 5} = \frac{1}{23}\m{0 & -100 & 105 & 0 & 4 }$

        $ x^*_{2, 4, 5} = \m{0 & 13 & 0 & -21 & 2 }$

        $ x^*_{3, 4, 5} = \frac{1}{19}\m{0 & 0 & 65 & -100 & 12 }$


    \item [15.10]
    \vspace{20mm}
    
    \item [16.2] \begin{enumerate}
        \item $$A= \m{3 & 1 & 0 &1  \\ 6 &2 &1&1}, b = \m{4 \\5}, c = \m{2 \\-1 \\-1 \\0 }$$
        \item The tableau is given by 
        $$ \m{ 3 & 1 & 0 & 1 & 4\\ 6 & 2 &1 & 1 & 5 \\ 2 & -1 & -1 &0 &0}$$

        we need to pivot about $(2, 3)$ and $(1, 4)$ and to get 

        $$ \m{ 3 & 1 & 0 & 1 & 4\\ 3 & 1 &1 & 0 & 1 \\ 5 & 0 & 0 &0 &1}$$

        \item  The basic feasible solution corresponding to this is $\m{0 \\ 0 \\ 1 \\ 4}$ with cost $-1$.
        \item The coefficients are $\m{5 & 0 & 0 & 0 }$.
        \item Yes it is optimal, as all the reduced row coefficients are positive.
        \item Yes, there is a basic feasible solution, since the phase I of the two-phase method returned a valid basic solution.
        \item $$\m{0 & 0 & -1 & 1 & 2 & -1 & 3 \\ 1 & \frac{1}{3} & \frac{1}{3} & 0 & -\frac{1}{3} & \frac{1}{3} & \frac{1}{3} \\ 2 & -1 & -1 & 0 & 0 &0 &0  }$$


    \end{enumerate}

    \item [16.3] 
        The tableau for this problem is given by $$\m{1 & 0 & 1 & 1 \\ 0 & 1 &1 & 2 \\ -1 & -1 & -3 & 0}$$
        By $R_3 + R_2 + R_1 \rightarrow R_3$ we can get the cannonical form.

        $$\m{1 & 0 & 1 & 1 \\ 0 & 1 &1 & 2 \\ 0 & 0 & -1 & 3}$$

        And then we pivot about $-1$ by $R_1 + R_3 \rightarrow R_3; R_2 - R_1 \rightarrow R_1$. 

        $$\m{1 & 0 & 1 & 1 \\ -1 & 1 & 0 & 1 \\ 0 & 1 & 0 & 5}$$

        Then the solution given is $\m{ 0 \\ 1 \\ 1}$ for a total cost of $5$.

    \item [16.4]
        We know there will be $3$ slack variables because we have three inequalities.
        The tableau for this problem is given by 
        $$\m{1 & 0 & 1 & 0 & 0 & 5 \\ 0 & 1 & 0 & 1 & 0 & 7 \\ 1 & 1 & 0 & 0 & 1 & 9\\ -2 & -1 & 0 &0 &0& 0  }$$

        There's a negative reduced cost coefficient, so we pivot along $(1, 1)$ via $R_3 -R_1 \rightarrow R_3$ and $R_4 + 2R_1 \rightarrow R_4$

        $$\m{1 & 0 & 1 & 0 & 0 & 5 \\ 0 & 1 & 0 & 1 & 0 & 7 \\ 0 & 1 & -1 & 0 & 1 & 4\\ 0 & -1 & 2 &0 &0& 10  }$$

        Again we pivot along $(3, 2)$ via $R_2 + R_4 \rightarrow R_2$ and $R_4 + R_3 \rightarrow R_4$

        $$\m{1 & 0 & 1 & 0 & 0 & 5 \\ 0 & 0 & 2 & 1 & 0 & 17 \\ 0 & 1 & -1 & 0 & 1 & 4\\ 0 & 0 & 1 &0 &1& 14  }$$

        Now our reduced row coefficients are all positive so our solution is $x = \m{5 \\ 4}$ for a total cost of 14.

    \item [16.5] \begin{enumerate}
        \item We can get the basis from the cannonical tableau 
        $$\m{ 0 & 1 & 1 & 2 \\ 1 & 0 & 3 & 4 }$$

        Then we can try to get the identity matrix on the right via $R_2 - 2R_1 \rightarrow R_2$ and $R_1 - R_2 \rightarrow R_2$ and $R_2 / 2 \rightarrow R_2$

        $$\m{-\frac{1}{2} & \frac{3}{2} & 0 & 1 \\ 1 & -2 & 1 & 0}$$
        
        so $B  = \frac{1}{2}\m{-1 & 3 \\ 2 & -4}$

        \item We know that $c_D^T = r_D^T + c_B^TB^{-1}D$ 
        $$ = [-1, 1] + [8, 7]^T \m{1 & 2 \\ 3 & 4} = [30, 47]$$ 
        so the missing part of $c = [30, 47]^T$.
        \item The basic feasible solution is given by $B^{-1}b = \m{1 & 2 \\ 3 & 4} \m{5 \\ 6} = \m{16 \\ 38}$.
        \item The first two elements are given above and the last element is given by $-c_B^TB^{-1}b = - \m{7 & 8}\m{16 \\ 38} = -426$
        The missing values are $[16, 38, −416]^T$.
        
    \end{enumerate}


    \item [16.6] The columns in the constraint matrix A corresponding to $x^+, x^-$
    are linearly dependent. Hence they
    cannot both enter a basis at the same time. This means that only one variable, can assume a
    nonnegative value; the nonbasic variable is necessarily zero.
    
    \item [16.8] \begin{enumerate}
        \item $\m{6 \\ 0 \\7 \\ 5\\0 }$ is the basic feasible solution for this tableau, with $f(x) = 8$
        \item $\m{ 0 & 4 & 0 &0 &-4}$
        \item Yes, as we can get any negative value since the last column is all negative.
        \item We would neeed to pivot about $(3,2)$ to get
        $$\m{0 & 0 & -\frac{1}{3} & 1 & 0 & \frac{8}{3} \\ 
             1 & 0 & -\frac{2}{3} & 0 & 0 & \frac{4}{3} \\
             0 & 1 & \frac{1}{3} & 0 & -1 & \frac{7}{3} \\
             0 & 0 & -\frac{4}{3} & 0 & 0 & -\frac{4}{3}
        }$$
        \item $x = [52, 0, 76, 28, 23]^T$
        \item Note from the tableau $a_2 = a_4 + 2a_1 + 3a_3$ and $ a_5 = -a_4 -2a_1  - 3a_3$ so therefore 
        $\m{2 \\ -1 \\ 3\\1\\0}, \m{-2 \\ 0 \\-3 \\-1\\-1} \in kernel(A)$. By RN theorem we know that $nullity(A) = 2$
        Since these two vectors are linearly independent, they form a  basis for the kernel of $A$.
    \end{enumerate}

    \item [16.9]
    \begin{enumerate}
        \item We can rewrite this as minimize $f(x) = x_1 + 2x_2$ s.t. $x_2 - s_1 = 1$
        \item First we create our tableau $$\m{ 0 & 1 & -1 & 1 \\ 1 & 2 & 0 & 0 }$$
        
        Then for phase 1 we try to minimize $f(x) = y_1$ 
        $$\m{ 0 & 1 & -1 & 1 & 1 \\ 0 & 0 & 0 & 1 & 0 }$$

        We zero out the bottom row via $ R_2 - R_1 \rightarrow R_2$
        $$\m{ 0 & 1 & -1 & 1 & 1 \\ 0 & -1 & 1 & 0 & -1 }$$

        Then we pivot about $(1, 2)$ via $R_2 + R_1 \rightarrow R_2$ and get back our original matrix
        $$\m{ 0 & 1 & -1 & 1 & 1 \\ 0 & 0 & 0 & 1 & 0 }$$

        The we start phase 2 by pivoting about $(1, 2)$ to get
        $$\m{ 0 & 1 & -1 &  1 \\ 1 & 0 & 2 & -2  }$$
        so our solution is $ x = \m{0 & 1}^T$.

    \end{enumerate}
        
    \item [16.10]
    \begin{enumerate}
        \item A basic solution is found when $x_2 = 0$ which forces $x_1= 1$ for $[1, 0 ]^T$.
        \item $\m{1 & -1 & 1}$
        \item The algorithm terminates since the problem is unbounded, as the element at $(1, 2) < 0$ 
        \item Note that any $x_1 \in \mathbb{R}$  the vector of the form $[x_1, x_1 -1]$ is feasible, so the objective function can take any arbritary value in $\mathbb{R}$.
    \end{enumerate}

    \item [16.11]
    First we construct the cannonical tableau $$\m{1 & 2 & -1 & 0 & 3 \\ 2 & 1 & 0 & -1  & 3 \\ 1 & 1 & 0 & 0 &0 }$$

    With $x_1, x_2$ as basic variables we get $ B = \m{1 & 2 \\ 2 & 1} \implies B^{-1} = \frac{1}{3}\m{-1 & 2 \\ -1 & 2}$

    so then $\lambda_d = \frac{1}{3}\m{1 & 1} \m{-1 & 2 \\ 2 & -1} = \m{\frac{1}{3} \\ \frac{1}{3}}$ and $r_d = c^T_d - \lambda^T D = 0 - \m{\frac{1}{3} \\ \frac{1}{3}} \m{-1 & 0 \\ 0 & -1 } = \m{\frac{1}{3} & \frac{1}{3}}$

    Since these are all non-negative, this is an optimal solution $\m{1 & 1}^T$ with cost 2.

    \item [16.12]
    \begin{enumerate}    
        \item The cannonical tableau has the form
        $$\m{5 & 1 & -1 & 0 & 0 & 11 \\
             -2 & -1 & 0 & 1 & 0 & -8 \\
             1 & 2 & 0 & 0 & -1 & 7 \\
             4 & 3 & 0 & 0 &0 & 0
        }$$

        There is no basic feasible solution, so we can add in $y_1, y_2, y_3$ to get an updated tableau

        $$\m{5 & 1 & -1 & 0 & 0 & 1 & 0 & 0 & 11 \\
             2 & 1 & 0 & -1 & 0 & 0 & 1 & 0& 8 \\
             1 & 2 & 0 & 0 & -1 & 0 & 0 &1 &7 \\
             0 & 0 & 0 & 0  &0 & 1 & 1 & 1 & 0
        }$$

        Our $B^{-1} = I_3, y_0 = \m{11 \\ 8 \\7}$ 

        with $y_1, y_2, y_3$ as our basic variables we get $\lambda = [1, 1, 1]^T$ and $r_D = [-8, -4, 1, 1, 1]^T$.
        So we pivot about $(1, 1)$ to bring $x_1$ into the basis.

        After this our $B^{-1} = \m{\frac{1}{5} & 0 & 0 \\ -\frac{2}{5} & 1 & 0 \\ -\frac{1}{5} & 0 & 1}, y_0 = \m{\frac{11}{5}, \frac{18}{5}, \frac{24}{5}}$

        with $x_1, y_2, y_3$ as our basis we get $\lambda = [-\frac{3}{5}, 1, 1]^T$ and $r_D = [-\frac{12}{5}, -\frac{3}{5}, 1, 1, \frac{8}{5}]^T$.

        So we pivot about $(2, 3)$ to bring $x_2$ into the basis.

        After this our $B^{-1} = \m{\frac{2}{9} & 0 & -\frac{1}{9} \\ -\frac{1}{3} & 1 & -\frac{1}{3} \\ -\frac{1}{9} & 0 & \frac{5}{9}}, y_0 = \m{\frac{5}{3}, 2, \frac{8}{3}}$

        with $x_1, x_2, y_3$ as our basis we get $\lambda = [-\frac{1}{3}, 1, -\frac{1}{3}]^T$ and $r_D = [-\frac{1}{3}, 1, -\frac{1}{3}, \frac{4}{3}, \frac{4}{3}]^T$.

        So we pivot about $(3, 2)$ to bring $x_3$ into the basis.

        After this our $B^{-1} = \m{0 & \frac{2}{3}  & -\frac{1}{3} \\ -1 & 3 & -1 \\  0 & -\frac{1}{3} & \frac{2}{3}}, y_0 = \m{3 \\ 6 \\2}$

        with $x_1, x_2, x_3$ as our basis we get $\lambda = 0$ and $r_D = [0, 0, 1, 1, 1]^T$.

        So our initial solution = $[3, 2, 6, 0, 0]^T$

        Going back to our original tableau with $x_1, x_2, x_3$ as our basis we get $\lambda = [0, \frac{5}{3}, \frac{2}{3}]$ which are the same ad $r_d > 0$

        so the optimal solution is $[3, 2]^T$.

        \item The cannonical tableau for this problem is 
         $$\m{1 & 2 & 1 & 2 & 1 & 0 & 0 & 20 \\
              6 & 5 & 3 & 2 & 0 & 1  & 0& 100 \\
              3 & 4 & 9 & 12 & 0 & 0&1 & 75 \\
              -6 & -4 & -7 & -5 & 0 & 0&0  &0 \\
               
         }$$

        Our $B^{-1} = I_3, y_0 = \m{20 \\ 100 \\75}$ 

        with $y_1, y_2, y_3$ as our basic variables we get $\lambda = 0$ and $r_D = [-6, -4, -7, -5]^T$.
        So we pivot about $(3, 2)$ to bring $x_2$ into the basis.

        After this our $B^{-1} = \m{1 & 0 & -\frac{1}{9} \\ 0 & 1 & -\frac{1}{3} \\ 0 & 0 & \frac{1}{9}}, y_0 = \m{\frac{35}{3}, 75, \frac{25}{3}}$

        with $x_1, x_2, y_3$ as our basis we get $\lambda = [0, 0, -\frac{7}{9}]^T$ and $r_D = [-\frac{11}{3}, -\frac{8}{9}, \frac{13}{3}, \frac{7}{9}]^T$.

        So we pivot about $(2, 1)$ to bring $x_1$ into the basis.

        After this our $B^{-1} = \m{0 & \frac{2}{3}  & -\frac{1}{3} \\ -1 & 3 & -1 \\  0 & -\frac{1}{3} & \frac{2}{3}}, y_0 = \m{3 \\ 6 \\2}$

        with $x_1, x_2, x_3$ as our basis we get $r_D = [\frac{27}{15}, \frac{43}{15}, \frac{11}{15}, \frac{8}{15}]^T$.

        so the optimal solution is $[15, 0, \frac{10}{3}, 0]^T$.

    \end{enumerate}

    \item [16.14]
    \begin{enumerate}
        \item $\m{1 \\ 0}, \m{0 \\ 2}$
        \item any vector in the $span(\m{2 \\ 1})$ will be a valid $c_1, c_2$.
        \item These are all 0 since the basic feasible solutions are optimal. 
    \end{enumerate}

\end{enumerate}

\end{document}