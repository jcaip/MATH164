\documentclass[10pt,a4paper]{article}
\addtolength{\oddsidemargin}{-.875in}
\addtolength{\evensidemargin}{-.875in}
\addtolength{\textwidth}{1.75in}
\addtolength{\topmargin}{-.875in}
\addtolength{\textheight}{1.75in}

\usepackage{amsmath,amssymb}
\DeclareMathOperator*{\E}{\mathbb{E}}
\DeclareMathOperator*{\R}{\mathbb{R}}
\DeclareMathOperator*{\Q}{\mathbb{Q}}
\DeclareMathOperator*{\N}{\mathbb{N}}
\DeclareMathOperator*{\I}{\mathbb{I}}
\DeclareMathOperator*{\argmax}{arg\,max}
\DeclareMathOperator*{\argmin}{arg\,min}
\usepackage{mathtools}
\DeclarePairedDelimiter{\ceil}{\lceil}{\rceil}
\DeclarePairedDelimiter{\abs}{\lvert}{\rvert}
\begin{document}

Jesse Cai

MATH 164 Homework 1

304634445

\begin{enumerate}
    \item \textbf{6.1 Deterime if the given point is definitely a local minimizer, definitely not, or possibly}
    \begin{enumerate}
        \item Definitely not a local minimizer, violates FONC $d = [1, 1]$ is feasible and $d^T\nabla f(x^*) = \sqrt{2} > 0$.
        \item Possibly a local minimizer, $x^*$ satisfies FONC.
        \item Definitely a local minimizer, this satisfies SOSC, as $x*$ is an interior point, $\nabla f(x^*) = 0$ and $F(x^*) = I([1, 2]^T)  = 3 > 0$.
        \item Definitely not a local minimizer, violates SONC $d = [0, 1]^T$ is feasible and $d^TF(x^*)d = -1 \leq 0$.
    \end{enumerate}

    \item \textbf{6.2 Show that if $x^*$ is a global minimizer over $\Omega$ and $x^* \in \Omega' \subset \Omega$ then $x^*$ is also a global minimizer over $\Omega'$}

        From Def 6.1, $\forall x \in \Omega, f(x) \geq f(x^*)$. Pick some element in $\Omega'$. $y \in \Omega' \implies y \in \Omega \implies f(y) \geq f(x^*)$.
        So $\forall y \in \Omega', f(y) \geq :(x^*) \implies x^*$ is a global minimizer over $\Omega'$.

    \item \textbf{6.4 Show $x_0 + \argmin_{x \in \Omega} f(x) = \argmin_{y \in \Omega'} f(y)$ }

        Let $x^* = \argmin_{x \in \Omega} f(x), y^* = \argmin_{y \in \Omega'} f(y)$.
        Suppose $x^* \neq y^* + x_0$. Then either $x^* \leq  y^* + x_0$ or $y^* + x_0 \leq x^*$.

        WLOG consider $x^* \leq  y^* + x_0 \implies x^* - x_0 \leq y^*, x^* - x_0\in \Omega'$, which is a contradiction, as we defined $y^*$ as the mininum. 

    \item \textbf{6.8 Consider $f(x) = x^T \left[\begin{matrix} 1& 2 \\4 & 7 \end{matrix}\right] x + x^T\left[ 3 \\5 \right] + 6$}
    \begin{enumerate}
        \item $\nabla f(x) = \left( \left[\begin{matrix} 1& 2 \\4 & 7 \end{matrix}\right] + \left[\begin{matrix} 1& 4 \\2 & 7 \end{matrix}\right] \right)
        \left[ \begin{matrix} 1 \\ 1 \end{matrix}\right] + \begin{bmatrix} 3 \\ 5 \end{bmatrix} = \begin{bmatrix} 11 \\ 25\end{bmatrix}$

        $F(x) = \begin{bmatrix} 2 &6 \\ 6 & 14 \end{bmatrix} $

        \item The unit vector in the direction of max increase is given by normalizing $\nabla f(x)$. 
        $D_d f(x) = \begin{bmatrix} \frac{11}{\sqrt{746}} \\ \frac{25}{\sqrt{746}} \end{bmatrix} \nabla f(x) = \sqrt{746}$

        \item Solving $\nabla f(x) = 0$ yields $x = \begin{bmatrix} 2 & 6 \\ 6 & 14 \end{bmatrix}^{-1} \begin{bmatrix} -3 \\ -5 \end{bmatrix} = \begin{bmatrix} \frac{3}{2}  \\ -1\end{bmatrix}$
        
        However, this point does not satisfy SONC as for $d = -x, d^TF(x)d = -0.5 < 0$.
    \end{enumerate}

    \item \textbf{6.11 Consider minimize $-x_2^2$ s.t. $\abs{x_2} \leq x_1^2, x_1 \geq 0$}
    \begin{enumerate}
        \item $\nabla f(x) = \begin{bmatrix} 0 \\ -2x_2 \end{bmatrix}, \nabla f(0) = 0$ so for any feasible direction d $d^T \nabla f(0) = 0$, so FONC is satisfied. 
        \item 0 is a local maxamizer but not a strict local maxamizer, as we can move along the $x_1$ axis for more maxima.
    \end{enumerate}

    \item \textbf{6.12 Consider minimizing $f(x) = 5x_2$ s.t. $x_1^2 + x_2 \geq 1$}
    \begin{enumerate}
        \item $\nabla f([0, 1]^T) = [0, 5]^T$. The set of feasible directions  is $ \{ [d_1, d_2] : d_1, d_2 > 0 \}$.
        But then $d^T\nabla f([0,1]^T) = 5d_2 > 0 $ because $d_2 > 0$. So the FONC is satisfied.

        \item Note that $F([0, 1]^T) = 0$ so $\forall d : d^T F(x) d = 0$ and SONC is satisfied.
        \item No $x^* = [0, 1]^T$ is not a local
    \end{enumerate}

    \item \textbf{6.16 Consider minimizing $f(x) = 4x_1^2 - x_2^2$ s.t. $x_1^2 - 2x_1 - x_2 \geq 0, x_1 \geq 0, x_2 \geq 0$}
    \begin{enumerate}
        \item $\nabla f(x) = [8x_1, -2x_2]^T $ so evaluated at $x^* = [0, 0]^T, \nabla f(x^*) = [0,0]^T$ and as such $\forall d : d^T \nabla f(x^*) = 0$ and FONC is satisfied.
        \item Note that $F = \begin{bmatrix} 8 & 0 \\ 0 & -2 \end{bmatrix}$ and the set of feasible directions is  $\{ [d_1, d_2]^T : d_2 \leq 2d_1, d_1 \geq 0, d_2 \geq 0 \}$
        
        Then we know that $d^T Fd = 8d_1^2 - 2d_2^2  \geq 8d_1^2 - 2(2d_1^2) =0$ so SONC is satisfied.
        
    \end{enumerate}

    \item \textbf{8.1 Perform two iterations of the minimization of $f(x_1, x_2) = x_1 + \frac{1}{2}x_2 + \frac{1}{2}x_1^2 + x_2^2 +3$}
    
        We'll rewrite this in standard form $f(x) = x^T\begin{bmatrix}\frac{1}{2} & 0 \\ 0 & 1\end{bmatrix}x + x^T[1, \frac{1}{2}] + 3$

        Then we know that $x_{n+1}  = x_n - \alpha_n \nabla f(x_n)$ and $\alpha_n = \frac{g_n^Tg_n}{g_n^TFg_n}$.
        
        $g_0 = \nabla f(x_0) = [1, \frac{1}{2}]^T$ so $x_1 = -\frac{5}{6}[1, \frac{1}{2}]^T = [-\frac{5}{6} ,  -\frac{5}{12}]^T$
        
        $g_1 = \nabla f(x_1) = [\frac{1}{6}, -\frac{1}{3}]^T$ so $x_2 = [-\frac{5}{6}, -\frac{5}{12}]^T -\frac{5}{9}[\frac{1}{6}, -\frac{1}{3}]^T = \begin{bmatrix}-\frac{25}{27} \\ -\frac{25}{108}\end{bmatrix}$
        
        Analytically the solution is found when $x^* = F^{-1} (-b) = \begin{bmatrix} 1& 0 \\ 0 &\frac{1}{2}\end{bmatrix} \begin{bmatrix}-1 \\ -\frac{1}{2}\end{bmatrix} = [-1, - \frac{1}{4} ]^T$
        
    \item \textbf{8.8 Find the largest range of values for $\alpha$ for which the algorithm is globally convergent.}
    
        First we'll rewrite into standard form: $f(x) = x^T \begin{bmatrix} 6 & 4 \\ 4 & 6\end{bmatrix}x + x^T\begin{bmatrix} 5 \\ 6\end{bmatrix} + 7$.

        We can take up to the smallest eigenvalue of $F$ as the step size, as taking a step larger than that will lead to divergence rather than convergence. 
        So we find the eigenvalues of $F$ by solving $f(\lambda) = \lambda^2 - 12\lambda +20$ to get $\lambda = 2, 10$. 

        The max step size is then $\frac{2}{\lambda_{max}}$ (Thm 8.3). So any $0 < \alpha < \frac{2}{10}$ will be gaurenteed to converge.
            
    \item \textbf{8.9 Consider the system of equations $h(x) = \begin{bmatrix} 4+3x_1+2x_2 \\ 1+2x_1+3x_2\end{bmatrix}$ }
    \begin{enumerate}
        \item Find when $h(x) = 0$. 
        
        We can rewrite $h(x)$ as a matrix multiplication $h(x) = \begin{bmatrix} 3 & 2 \\ 2 & 3\end{bmatrix} \begin{bmatrix} x_1 \\x_2 \end{bmatrix} = \begin{bmatrix} -4 \\ -1 \end{bmatrix}$
        Then by multiplying by the inverse matrix we get $\begin{bmatrix} x_1 \\ x_2 \end{bmatrix} = \frac{1}{5}\begin{bmatrix}3 & -2 \\ -2 & 3 \end{bmatrix}\begin{bmatrix} -4 \\ -1 \end{bmatrix} = \begin{bmatrix} -2 \\ 1 \end{bmatrix}$.

    \end{enumerate}

    \item \textbf{8.10 Consider the function $f(x) = \frac{3}{2}(x_1^2 +x_2^2) + (1+a)x_1x_2 - (x_1 + x_2) + b$}
    \begin{enumerate}
    \item $f(x) = \frac{1}{2}x^T \begin{bmatrix} 3 & 1+a \\ 1+a & 3 \end{bmatrix}x - x^T \begin{bmatrix} 1 \\ 1 \end{bmatrix} + b$
    \item We can solve analytically for $x^*$ as $x^* = Q^{-1}b = \frac{1}{9 - (1+a)^2} \begin{bmatrix} 3 & -a-1 \\ -a-1 & 3 \end{bmatrix} \begin{bmatrix} 1 \\ 1\end{bmatrix}$
    
    We need the determininet to be $>0$ for positive definitite, so $-4 \leq a \leq 2$. The minimia always exists for any $b$.
    \item If our alpha is $\frac{2}{5}$ then our max eigenvalue is 5. Solving for this yields $-3 \leq a \leq 1 $.
    \end{enumerate}

    \item \textbf{8.11 Consider $f(x) = \frac{1}{2} (x-c)^2$, minimized using an iterative algoritm}
    \begin{enumerate}
        \item $$f(x^{k+1}) = \frac{1}{2}(x^k - \alpha_k f'(x^k) - c)^2$$
        $$= \frac{1}{2}(x^k -c - \alpha_k f'(x^k))^2 = \frac{1}{2}[(x^k -c)^2 -2\alpha_k f'(x^k)(x^k -c) + (\alpha_k f'(x^k))^2 ]$$
        $$= f(x^k) - \alpha_k f'(x^k)(x^k -c) + \frac{1}{2}(\alpha_k f'(x^k))^2$$

        Recall that $f'(x^k) = x^k - c$ so plugging in yields
        $$= f(x^k)  - \frac{1}{2}\alpha_k^2 (x^k - c)^2 = f(x^k)(1 -\alpha_k)^2 = f(x^{k+1})$$

        \item Note that from a we get $f(x^k) = f(x^0) \prod_{i=0}^{k-1} (1 - \alpha_i)^2$
        
        So for this to converge $\prod_{i=0}^{\infty} (1 - \alpha_i)^2$ must go to 0.
        Since $ 0 < \alpha_i < 1$ this is the same as $\prod_{i=0}^{\infty} (1 - \alpha_i) = 0 \iff \sum_{k=0}^\infty \alpha_k = \infty$

    \end{enumerate}

    \item \textbf{8.12 Consider minimizing $x^3 - x$}
   
    We can rewrite $f(x) = x(x+1)(x-1)$ and $\nabla f(x) = 3x^2 -1 $ so we know that the minima occurs at $\frac{1}{\sqrt 3}$.

    \item \textbf{8.15 Consider minimizing  $\Vert ax-b \Vert^2$}
    \begin{enumerate}
        \item We can rewrite this as $f(x) = (ax-b)^T(ax-b) = (ax^T-b^T)ax - (ax^T-b^T)b = \Vert a \Vert^2 x^2 - 2a^Tb x + \Vert b \Vert^2$. 
        
        The minimia can be found when $\nabla f(x) = 0 \implies 2 \Vert a \Vert^2 x - 2 a^Tb= 0 \implies x^* = \frac{a^Tb}{\Vert a \Vert^2}$

        \item We can use Thm 8.3 to get that the max step size should be $\frac{2}{2 \Vert a \Vert^2}  = \frac{1}{\Vert a \Vert^2}$.
    \end{enumerate}

    \item \textbf{8.16 Consider minimizing  $\Vert Ax-b \Vert^2$}
    \begin{enumerate}
        \item We can rewrite this as $$f(x) = (Ax-b)^T(Ax-b) = (x^TA^T-b^T)Ax - (x^TA^T-b^T)b = $$
        $$= x^TA^TAx -b^TAx - x^TA^Tb +b^Tb  = x^TA^TAx -2b^TA x +b^Tb  $$

        $\nabla f(x) = 2A^TAx - 2b^TA$ and $Q = 2A^TA$.

        \item $$x^{k+1} = x^k - \alpha (\nabla f(x^k)) = x^k - \alpha 2 (A^TAx - b^TA)  $$
        
        \item For $A = \begin{bmatrix} 1 & 0 \\ 0 & 2 \end{bmatrix}$ $A^TA = \begin{bmatrix} 1 & 0  \\ 0 & 4\end{bmatrix}$ and the eigenvalues of $A^TA$  are 1, 4 respectively. 
        
        So by Thm 8.3 this corresponds to a max step size of $\frac{1}{4}$.
        
    \end{enumerate}
   
    \item \textbf{8.18 Show that steepest descent converges in one step iff $\nabla f(x^0)$ is an eigenvector.}
    
    For steepest descent to converge in one step 

    $$x^1 = x^* = Q^{-1}b = x^0 + \alpha g^0 \implies b = Qx^0 + \alpha Qg^0 \implies Qx^0 - b = - \alpha Qg^0 $$

    But $Qx^0 -b = g^0$ so $\frac{1}{\alpha}g^0 =  Qg^0 \implies g^0$ is an eigenvector.

    Suppose $g^0$ was an eigenvector. Then we know $Qg^0 = \lambda g^0$ with $\lambda = \frac{1}{\alpha}$ (from above).

    $$x^1  = x^0 - \alpha g^0 \implies Qx^1 = Q(x^0 - \alpha g^0) \implies Qx^1 = Qx^0 -  \alpha Q g^0 \implies Qx^1 = Qx^0 - g^0 = b $$
    but this means that $x^1 = Q^{-1}b = x^*$.
    
    \item \textbf{8.21 Find the largest step size such that the algorithm is globally convergent}
    \begin{enumerate}
        \item We can find the eigenvalues of the hessian $Q = \begin{bmatrix}6 & 4 \\ 4 & 6 \end{bmatrix}$. The max eigevalue is 10 so the max step size is $\frac{1}{5}$.
        \item This is the same hessian as above $A+A^T = Q = \begin{bmatrix}6 & 4 \\ 4 & 6 \end{bmatrix}$. The max eigevalue is 10 so the max step size is $\frac{1}{5}$.
    \end{enumerate}

\end{enumerate}
\end{document}